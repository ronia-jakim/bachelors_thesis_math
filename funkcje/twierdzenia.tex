\usepackage{amsthm}
\usepackage{thmtools}

\declaretheoremstyle[
    headfont=\bfseries\color{green!80!black},
    bodyfont=\normalfont,
    notebraces={: }{},
    notefont=\color{green},
    headformat=\NAME$\;$\NUMBER\NOTE,
    postheadspace=\newline,
    mdframed={
        linewidth=2pt,
        rightline=false, topline=false, bottomline=false,
        linecolor=green, backgroundcolor=yellow!4,
        skipabove=3mm, skipbelow=3mm
    }
]{defbox}

\declaretheoremstyle[
    headfont=\bfseries\color{orange!80!black},
    bodyfont=\normalfont,
    notebraces={: }{},
    notefont=\color{orange},
    headformat=\NAME$\;$\NUMBER\NOTE,
    postheadspace=\newline,
    mdframed={
        linewidth=2pt,
        rightline=false, topline=false, bottomline=false,
        linecolor=orange, backgroundcolor=yellow!4,
        skipabove=3mm, skipbelow=3mm
    }
]{theorembox}

\newtheoremstyle{straightstyle}
  {6pt} % Space above
  {6pt} % Space below
  {} % Body font
  {} % Indent amount
  {\bfseries} % Theorem head font
  {.} % Punctuation after theorem head
  {.5em} % Space after theorem head
  {} % Theorem head spec (can be left empty, meaning `normal')

\newtheoremstyle{italicsstyle}
  {6pt} % Space above
  {6pt} % Space below
  {\slshape} % Body font
  {} % Indent amount
  {\bfseries} % Theorem head font
  {.} % Punctuation after theorem head
  {.5em} % Space after theorem head
  {} % Theorem head spec (can be left empty, meaning `normal')

\declaretheorem[ %
  name=Definition, %
  numberwithin=section, %
  style=defbox
]{definition}

\declaretheorem[ %
  name=Theorem, %
  numberwithin=section, %
  style=theorembox
]{theorem}

\declaretheorem[ %
  name=Proposition, %
  numberlike=theorem, %
  style=theorembox
]{proposition}

\declaretheorem[ %
  name=Corollary, %
  numberlike=theorem, %
  style=straightstyle
]{corollary}

\declaretheorem[ %
  name=Lemma, %
  numberlike=theorem, %
  style=straightstyle
]{lemma}

\declaretheorem[ %
  name=Remark, %
  numberlike=definition, %
  style=italicsstyle
]{remark}

\declaretheorem[ %
  name=Example, %
  numberwithin=section, %
  style=straightstyle
]{example}


% \renewenvironment{proof}{{\bfseries Proof}$ $\newline}{
%   \begin{flushright} $ \spadesuit $ \end{flushright}$ $\newline
% }

\usepackage{cleveref}

%\crefname{definition}{definicja}{definicje}
%\Crefname{definition}{Definicja}{Definicje}
%
%\crefname{theorem}{twierdzenie}{twierdzenia}
%\Crefname{theorem}{Twierdzenie}{Twierdzenia}
%
%\crefname{lemma}{lemat}{lematy}
%\Crefname{lemma}{Lemat}{Lematy}
%
%\crefname{remark}{uwaga}{uwagi}
%\Crefname{remark}{Uwaga}{Uwagi}
