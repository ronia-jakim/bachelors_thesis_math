We can think of coloring diagrams with a chosen palette $(R, M, \mathcal{C}_\pm)$ as being a {\color{red}functor from a set of diagrams to a set of matrices}.

\begin{definition}[color checking matrix]\label{def:color checking matrix}
  Assigning segments of diagram $D$ to coordinates in $M^s$ and crossings to coordinates in $N_\pm^x$ it is possible to define a linear homomorphism $D\phi:M^s\to N_\pm^x$  as
  $$D\phi(m_1,...,m_s)=(\phi_\pm(\pi_{x_1}(m_1,...,m_s)), \phi_\pm(\pi_{x_2}(m_1,...,m_s)),...).$$
  Matrix that is created after choosing a basis for $M^s$ and $N_\pm^x$ will be called a \buff{color checking matrix}.
\end{definition}

\begin{proposition}
  Coloring $(m_1,...,m_s)\in M^s$ is admissible $\iff$ $(m_1,...,m_s)\in\ker D\phi$.
\end{proposition}

\begin{proof}
  We start by saying that 
  $$(m_1,..., m_s)\in\ker D\phi\iff [(\forall\;x_j\text{ crossing})\;\phi_\pm(\pi_{x_j}(m_1,..., m_s))=0].$$
  Which is to say that every coordinate of $D\phi(m_1,..., m_s)$ is zero. Proposition \cref{proposition male kernel kolorowania} says that it is equivalent with $(m_1,..., m_s)$ being an admissible coloring.
\end{proof}

We want to define which color checking matrices are equivalent. We will say that $D\phi$ and $D'\phi$ are equivalent if 
\begin{enumerate}
  \item they differ by a permutation of rows or columns, 
  \item one can be obtained from the other by adding a linear combination of rows or columns to another row or columns 
  \item one can be obtained from the other by adding a new row and a new column with only $0$ save for the term on their intersection, which is a unit.
\end{enumerate}
The first two points mean that two color checking matrices $D\phi, D'\phi:M^s\to N^x$ are equivalent if there exist an isomorphisms $\theta:M^s\to M^s$ and $\psi:N^x\to N^x$ such that
$$
\begin{tikzcd}[column sep=large]
  M^s\arrow[r, "D\phi"]\arrow[d, "\theta" left] & N^x\arrow[d, "\psi" right]\\ 
  M^s\arrow[r, "D'\phi" below] & N^x
\end{tikzcd}
$$
is a commutative diagram. 

In the most basic sense, two diagrams $D$ and $D'$ are isomorphic if there exists an isotopy $h_t:\R^2\to \R^2$ such that $h_0(D)=D$ and $h_1(D)=D'$ and $D'$ has crossings identical to those of $D$.

\begin{lemma}
  Isomorphic diagrams $D\sim D'$ yield isomorphic color checking matrices $D\phi\sim D'\phi$.
\end{lemma}

\begin{proof}
  In terms of color checking matrices, an isomorphism of diagrams defined above only relabels segments (permutes columns) and crossings (permutes rows).
\end{proof}

In reality, the isomorphism between diagrams which relates diagrams of the same knot is much more subtle. This relation $R$ is generated by the Reidemeister moves (see \cref{reidemeister-generating}). There is an induced relation on the set of color checking matrices, called $D(R)$, that relates color checking matrices of the same knot. For suitably chosen palettes, an invariant of the equivalent class of matrices of the same knot is easily obtainable. One of such palettes is the \buff{Alexander palette} {\boldmath$(R=\Z[\Z], M=\Z[\Z], \mathcal{C}_\pm)$}, where $\mathcal{C}_\pm$ are defined by homomorphisms
$$\phi_+(u,i,o)=(1-t)u+ti-o$$
$$\phi_-(u,i,o)=(1-t^{-1})u+t^{-1}i-o,$$
with $u$, $i$ and $o$ defined in \cref{crossing_type}. For the Alexander palette said invariant is e.g. the determinant of a color checking matrix up to multiplication by a unit. 

The following example illustrates the importance of choosing a suitable palette.

\begin{example}
  Consider a coloring of trefoil knot $3_1$ with palette {\color{red} podmienic na bardziej brzydka $(\Z, \Z, \phi_\pm(u,i,o)=2u-i+o)$}. The color checking matrix of its diagram with $3$ crossings is
  \begin{center}
    \begin{tikzpicture}
      \begin{knot}[
        consider self intersections, 
        clip width = 20pt, 
        % draft mode=crossings, 
        flip crossing=2
        ] 
        \strand[->, thick] (90:2) to[out=0, in=-60, looseness=2]
        (210:2) to[out=120, in=60, looseness=2]
        (-30:2) to[out=-120, in=180, looseness=2] 
        (90:2);
      \end{knot}

      \node at (6, 0) {
        $\det
        \begin{bmatrix}
          2 & -1 & 1\\ 
          -1 & 1 & 2 \\ 
          1 & 2 & -1 
        \end{bmatrix}=-3
      $};
    \end{tikzpicture}
  \end{center}

  while after the first Reidemeister move, the color checking matrix is

  \begin{center}
    \begin{tikzpicture}
      \begin{knot}[
        consider self intersections, 
        clip width = 20pt,
        %ignore endpoint intersections=false,
        % draft mode=crossings,
        % flip crossin=1,
        % flip crossing=2
        ] 
        \strand[<-, thick] (80:3) to[out=-100, in=30] 
        (90:2) to[out=-30, in=-60, looseness=2]
        (210:2) to[out=120, in=60, looseness=2]
        (-30:2) to[out=-120, in=210, looseness=2] 
        (90:2) to[out=150, in=-80] 
        (100:3) to[out=100, in=80, looseness=2] 
        (80:3);
      \end{knot}
      \fill[white] (90:2) circle (10pt);
      \draw[thick] ($(90:2)+({0.3*cos(210)}, {0.3*sin(210)})+(-0.02, -0.07)$) to[out=30, in=-120] ($(90:2) + ({0.3*cos(30}, {0.3*sin(30})+(0, 0.08)$);%(80:3);
      % \draw[->, thin] (90:2) to[out=-30, in=-60, loosenes=2] (210:2);
      \node at (6, 0) {$\det
          \begin{bmatrix}
            0 & 2 & 1 & -1 \\ 
            0 & -1 & 2 & 1 \\ 
            1 & 0 & -1 & 2\\ 
            1 & 1 & 0 & 0
          \end{bmatrix}=-8$
        };
    \end{tikzpicture}
  \end{center}
\end{example}

In order to ensure that the equivalence classes of color checking matrices under the relation induced by Reidemeister moves on diagrams have a known invariant, rules must be imposed on palettes. A particular set of palettes that will satisfy conditions to be mention are the Alexander palette and all palettes derived from it by either a ring homomorphism or a module homomorphism.
\begin{enumerate}
  \item To begin with, the coloring rule modules $\mathcal{C}_\pm$ are to be isomorphic to $M^2$, with the red arrow being an isomorphism
    $$
    \begin{tikzcd}
      M^2 \arrow[r, hookrightarrow] \arrow[rr, bend right=25, red, "\cong" below] & M^3 \arrow[twoheadrightarrow, r] & \mathcal{C}_\pm.
    \end{tikzcd}
    $$
    We want the red isomorphism to be $(u, i)\mapsto (u, i, \phi_\pm'(u, i))\in\mathcal{C}_\pm$, with segments labeled like in \cref{crossing_type}, meaning that $c$ and $\gamma$ in \eqref{phi equations1} and \eqref{phi equations2} respectively are units. For the sake of convenience, take $c=\gamma=-1$.

    This property of palettes will be called a \buff{propagation rule} as knowing colors of two of the three segments allows one to calculate the color assigned to the remaining segment.
  \item The first Reidemeister move requires that
    $$a=1-b$$
    $$\alpha=1-\beta,$$
    where the variables are coefficients from \eqref{phi equations1} and \eqref{phi equations2}. This will be explained in greater detail in the next section. 
  \item Similarly, the second Reidemeister move requires
    $$\begin{cases}
      a\beta+\alpha=0\\ 
      \beta b=1.
    \end{cases}$$
    The reasoning behind this restriction, once again, will be elaborated on in the next section.
\end{enumerate}

% {\color{red}from now on we want to work with palettes that satisfy the following conditions: dwuwymiaroowe C i propagation rule, to a = 1-b, b beta = 1 oraz a beta + alpha = 0. in particular, we want to work with the Alexander palette and all palettes derived from it}







