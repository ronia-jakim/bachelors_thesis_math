\hl{An oriented diagram $D$ of knot $K$ has two types of crossings, pictured in} \cref{crossing_type}. A diagram coloring, in essence, is an assignment of values from some mathematical object (i.e. $R$-module) to segments of $D$.

\begin{figure}[h]\centering
  \begin{tikzpicture}[
      pics/arrow/.style={code={%
        \draw[line width=0pt,{Computer Modern Rightarrow[line width=1pt,width=2ex,length=1.5ex]}-] (-0.5ex,0) -- (0.5ex,0);
        }
      },
      pics/arrow2/.style={code={%
        \draw[line width=0pt,-{Computer Modern Rightarrow[line width=1pt,width=2ex,length=1.5ex]}] (-0.5ex,0) -- (0.5ex,0);
        }
      }
    ]
    % \draw[very thick, ->] (0,0) node[below] {i} -- (1.5, 2) node[above] {o};
    % \fill[white] (0.75, 1) circle (6pt);
    % \draw[very thick, ->] (1.5, 0)--(0, 2) node[above] {u};
    % \node at (0.75, -0.5) {$+$};
    %
    % \draw[very thick, ->] (4.5, 0) node[below] {i} --(3, 2) node[above] {o};
    % \fill[white] (3.75, 1) circle (6pt);
    % \draw[very thick, ->] (3, 0)--(4.5, 2) node[above] {u};
    % \node at (3.75, -0.5) {$-$};

    \begin{knot}[
      consider self intersections, 
      clip width =10pt
      ]
      \strand[ultra thick] (0, 0) to[out=90, in=180]
      (2, 2) to[out=0, in=180] pic[pos=0,sloped]{arrow}
      (6, -2) to[out=0, in=180]  pic[pos=0,sloped]{arrow}
      (10, 2) to[out=0, in=90]  pic[pos=0,sloped]{arrow}
      (12, 0) to[out=-90, in=0]
      (10, -2) to[out=180, in=0] 
      (6, 2) to[out=180, in=0] pic[pos=0,sloped]{arrow2}
      (2, -2) to[out=180, in=-90]
      (0,0);
    \end{knot}
    \draw[dashed, thick] (4, 0) circle (30pt);
    \draw[dashed, thick] (8, 0) circle (30pt);

    \node at (8, -1.3) {$-$};
    \node at (4, -1.3) {$+$};
    
    \node at ($(4, 0)+(.6, -.6)$) {$u$};
    \node at ($(4, 0)+(.6, .6)$) {$i$};
    \node at ($(4, 0)+(-.6, -.6)$) {$o$};

    \node at ($(8, 0)+(.6, -.6)$) {$o$};
    \node at ($(8, 0)+(.6, .6)$) {$u$};
    \node at ($(8, 0)+(-.6, .6)$) {$i$};
  \end{tikzpicture}
  \caption{Two types of crossing in oriented diagram.\label{crossing_type}}
\end{figure}

\begin{definition}[palette]\label{def paleta}
  We say that a quadruple {\boldmath$(R, M, \mathcal{C}_\pm)$} is a \buff{palette} if $R$ is a commutative ring with unity, $M$ an $R$-module and $\mathcal{C}_\pm$ are two $R$-modules, corresponding to the two types of crossings (\cref{crossing_type}) such that $\mathcal{C}_\pm\subseteq M^3$.
\end{definition}

If a palette $(R, M, \mathcal{C}_\pm)$ is given along with a ring homomorphism $f:R\to S$, then the image of this palette through induced palette homomorphism $f_*$ is $(S, M\otimes_R S, \mathcal{C}_\pm\otimes_R S)$. Similarly, for an $R$-module homomorphism $g:M\to N$ we write $g_*(R, M, \mathcal{C}_\pm)=(R, N, g(\mathcal{C})_\pm)$. 

We will cumulatively call the two modules $\mathcal{C}_\pm$ the \buff{coloring rule} of palette $(R, M, \mathcal{C}_\pm)$ as they determine whether a coloring is admissible.

\begin{definition}[diagram coloring]
  A \buff{coloring of diagram} $D$ with $s$ segments and $x$ crossings (for knots $s=x$ \cref{ilosc segmentow to ilosc skrzyzowan}) is any element $(m_1,..., m_s)\in M^s$ that assigns elements of $M$ to each arc. 

  We will call a coloring \buff{admissible} if for every crossing $x_j$ of type $\pm$ we have 
  $$\pi_{x_j}(m_1,..., m_s)\in \mathcal{C}_\pm,$$
\hl{where $\pi_{x_j}:M^s\to M^3$ is a projection of module $M^s$ to the $M^3$ factor that corresponds to segments that constitute $x_j$.} 
\end{definition}

% Sometimes, a crossing is actually composed of only two segments, as is the case e.g. in the first Reidemeister move (\cref{reidemeister-generating}). For such a crossing $x_j$ we require that $\im(\pi_{x_j})\cong M^2\subseteq M^3$.

% Usually, a crossing comprises of exactly three segments. However, as is the case for the first Reidemeister move, a crossing can be comprised of only two segments. Thus, if $x_j$ is such a two-segment crossing, $\im(\pi_{x_j})$ should be isomorphic to $\{(x, y, y)\;:\;x,y\in M\}$.

We can now define two module homomorphisms
$$\phi_\pm:M^3\to M^3/\mathcal{C}_\pm=N_\pm$$
that take in as arguments the arcs constituting a crossing. Assuming that $M^3/\mathcal{C}_\pm\cong M$ (reasoning behind this assumption will be given in \cref{section 3.2}), we will take 
\begin{align}
  \phi_+(u,i,o)=au+bi+co \label{phi equations1} \\ 
  \phi_-(u,i,o)=\alpha u+\beta i+\gamma o \label{phi equations2}
\end{align}
for $u$, $i$, $o$ understood like in \cref{crossing_type} and with coefficients being homomorphisms $M\to M$.

\begin{lemma}\label{proposition male kernel kolorowania}
  A coloring $(m_1,..., m_s)\in M^s$ is a admissible $\iff$ for each crossing $x_j$ of type $\pm$ 
  $$\phi_\pm(\pi_{x_j}(m_1,...,m_s))=0.$$
\end{lemma}

\begin{proof}
  Stems from the fact that $\mathcal{C}_\pm=\ker\phi_\pm$.
\end{proof}



% \begin{definition}[category of palettes]
%   The set of palettes with morphisms induced by ring and module homomorphisms as above is a category of palettes.
% \end{definition}
