
\medskip

{\color{red}
In this chapter we will give a homological motivation to a more combinatorical invariant}

\begin{definition}[Alexander module]
  Given a group $G$, the abelianization of the commutator of a group $G$, $K_G^{ab}$, is called the \buff{Alexander module} of $G$. If $G$ is a knot group, then it is the Alexander module of the knot $K$
\end{definition}

Take $G=\langle G\;|\;R\rangle$ to be the Wirtinger presentation of $G$ obtained from diagram $D$. Because $K$ is a knot and not a link, we know that the number of segments is equal to the number of crossings, thus we can take $n=s=x$.

The group $G$ has $n$ generators and $n$ relations, and therefore the module $K^{ab}$ will have $(n-1)$ generators and $n$ relations still, as one of $G$ generators is lost due to abelianization. We start writing the resolution of $K_G^{ab}$ as follows:
\begin{center}
  \begin{tikzcd}
    ...\arrow[r] & \Z[\Z]^{n}\arrow[r, "A_D"] & \Z[\Z]^{n-1}\arrow[r] & K_G^{ab}\arrow[r] & 0
  \end{tikzcd}
\end{center}

\begin{definition}[Alexander matrix]
  The matrix of homomorphism $A_D$ in the diagram above is called the \buff{Alexander matrix} of group $G$ (knot $K$).
\end{definition}


\begin{proposition}
  Let $G$ be a knot group of $K$. Then it always has a resolution
  \begin{center}
    \begin{tikzcd}
      0\arrow[r] & \Z[\Z]^1 \arrow[r] & \Z[\Z]^{n}\arrow[r] & \Z[\Z]^{n-1}\arrow[r] & K_G^{ab}\arrow[r] & 0
    \end{tikzcd}
  \end{center}
  where $n$ is the number of crossings of the chosen diagram $D$ of knot $K$.
\end{proposition}

\begin{proof}
  Take $R=\Z[\Z]$ and consider its ring of fractions $R^{-1}R$. There is an homomorphism $R\to R^{-1}R$ which allows us to change the coefficients in the resolution of $K_G^{ab}$ as follows:
  \begin{center}
    \begin{tikzcd}
      0\arrow[r] & R^a\otimes_R R^{-1}R \arrow[r] & R^{n}\otimes_R R^{-1}R \arrow[r] & R^{n-1}\otimes_R \arrow[dll, rounded corners, to path={--([xshift=2ex]\tikztostart.east) |- ([xshift=-3ex, yshift=-2ex]\tikztostart.south) -| ([xshift=-2ex]\tikztotarget.west) -- (\tikztotarget)
      }]\\ 
      & K_G^{ab}\otimes_R R^{-1}R \arrow[r] & 0
    \end{tikzcd}
  \end{center}
  In \cref{prop: modul alexandera jest torsyjny} it was proven that $K_G^{ab}$ is a torsion module and so $K_G^{ab}\otimes_R R^{-1}R=0$. This reduces the sequence given above to a short exact sequence of vector spaces over the field $R^{-1}R$ which implies that 
  $$n=(n-1)+a\implies a=1.$$
\end{proof}


