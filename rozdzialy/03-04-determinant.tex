The reasoning presented in \cref{homological coloring} points at determinant of the coloring matrix being an invariant as was the case for the Alexander matrix. However, at the very moment the color checking matrix is not a knot invariant nor is its determinant. Any module $\mathcal{C}$ and associated with it pair of homomorphisms $\phi$ does not necessarily yield a "nice" coloring. The following example justifies the necessity of imposing restrictions to which a coloring rule must conform in order to be considered in the latter part of this paper.

\begin{example}
  Consider a coloring of trefoil knot $3_1$ with $\Z$ over the ring $\Z$ with $\phi_\pm(u, i, o)=2u-i+o$. The color checking matrix of its diagram with $3$ crossings is
  \begin{center}
    \begin{tikzpicture}
      \begin{knot}[
        consider self intersections, 
        clip width = 20pt, 
        % draft mode=crossings, 
        flip crossing=2
        ] 
        \strand[->, thick] (90:2) to[out=0, in=-60, looseness=2]
        (210:2) to[out=120, in=60, looseness=2]
        (-30:2) to[out=-120, in=180, looseness=2] 
        (90:2);
      \end{knot}

      \node at (6, 0) {
        $\det
        \begin{bmatrix}
          2 & -1 & 1\\ 
          -1 & 1 & 2 \\ 
          1 & 2 & -1 
        \end{bmatrix}=-3
      $};
    \end{tikzpicture}
    \begin{tikzpicture}
      \begin{knot}[
        consider self intersections, 
        clip width = 20pt,
        %ignore endpoint intersections=false,
        % draft mode=crossings,
        % flip crossin=1,
        % flip crossing=2
        ] 
        \strand[<-, thick] (80:3) to[out=-100, in=30] 
        (90:2) to[out=-30, in=-60, looseness=2]
        (210:2) to[out=120, in=60, looseness=2]
        (-30:2) to[out=-120, in=210, looseness=2] 
        (90:2) to[out=150, in=-80] 
        (100:3) to[out=100, in=80, looseness=2] 
        (80:3);
      \end{knot}
      \fill[white] (90:2) circle (10pt);
      \draw[thick] ($(90:2)+({0.3*cos(210)}, {0.3*sin(210)})+(-0.02, -0.07)$) to[out=30, in=-120] ($(90:2) + ({0.3*cos(30}, {0.3*sin(30})+(0, 0.08)$);%(80:3);
      % \draw[->, thin] (90:2) to[out=-30, in=-60, loosenes=2] (210:2);
      \node at (6, 0) {$\det
          \begin{bmatrix}
            0 & 2 & 1 & -1 \\ 
            0 & -1 & 2 & 1 \\ 
            1 & 0 & -1 & 2\\ 
            1 & 1 & 0 & 0
          \end{bmatrix}=-8$
        };
    \end{tikzpicture}
  \end{center}
\end{example}

The most important condition that $\mathcal{C}_\pm$ must meet is to be two dimensional. This will allow for propagation of coloring, meaning that knowing colors of two segments creating a crossing the third one can be calculated from $\phi_\pm$.

The following diagram
% \begin{center}
% \begin{tikzcd}
%   M^2 & M^3\arrow[l, twoheadrightarrow] \arrow[r, hookleftarrow] & \mathcal{C}\arrow[ll, bend left=20, red, "\sim"]
% \end{tikzcd}
% \end{center}
\begin{center}
\begin{tikzcd}
  M^2 \arrow[r, hookrightarrow] \arrow[rr, bend right=20, red, "\sim" below] & M^3 \arrow[r, twoheadrightarrow] & \mathcal{C}
\end{tikzcd}
\end{center}
must commute, with the red arrow being
$$(u, i)\mapsto (u, i, \phi_\pm'(u, i))$$
where $\phi_\pm'$ calculates the "out" segment in admissible coloring of each crossing (compare \cref{crossing_type}). Using (\ref{phi equations1}) and (\ref{phi equations2}) we can take $c$ and $\gamma$ to be any invertible elements, i.e. $c=\gamma=-1$, to have 
$$\phi_+'(u,i)=au+bi$$
$$\phi_-'(u,i)=\alpha u+\beta i$$

Notice that now a diagram with all but one segments colored can be easily colored in its entirety, using $\phi_\pm'$ on the crossing where the remaining segment starts.

