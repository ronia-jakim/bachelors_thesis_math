\section*{Plan działania}

\begin{enumerate}
  \item Relacje na macierzach -> Reidemeister
    \begin{enumerate}
      \item propagation rule - funkcja $\phi$, potencjalnie dla uproszczenia będziemy pisać $\phi_+$ i $\phi_-$ na reguły kolorowania dwóch typów skrzyżowania
      \item Diagram, s łuczków i x skrzyżowań - macierz która bardzo nie jest niezmiennikiem węzła, a zależy od diagramu.
      \item Wprowadzamy relację na zorientowanych diagramach (chociaż w sumie chyba nie potrzebuję orientacji, ale na takich pracuję więc elo)
    \end{enumerate}
  \item Smith normal form 
  \item Skein relations
  \item moduł Alexandera $6_1$ i $9{46}$, czy są różne
  \item rezolwenty
  \item zmiana pidów
\end{enumerate}
