\section{Introduction}

{\large\color{red}KTO CO ROBIŁ}

In mathematical terms, a knot is a particular embedding $S^1\hookrightarrow S^3$. A knot diagram is a projection $D:S^1\twoheadrightarrow \R^2$ along a vector such that no three points of the knot lay on this vector \cite{likorish-diagram}.

$S^1$ is an orientable space thus we can choose an orientation for a knot being considered. Then a diagram $D$ is oriented if it is a projection of an oriented $S^1$.

Intuitively, two knots $K_1$ and $K_2$ are equivalent if we can deform one into the other without cutting it and only manipulating it with our hands \cite{murasagi-equivalence}. This translates to equivalence of diagrams, which is generated by a set of moves, called the \buff{Reidemeister moves}. In the case of a diagram without an orientation, three moves are sufficient. When an orientation is imposed on $D$, at least $4$ moves are required \cite{ruchy_zorientowane}.

\section{What is a knot coloring}

Let $K$ be a knot and $D$ be its oriented diagram with $s$ segments and $x$ crossings. In such diagrams we can see two different crossing types as seen in \cref{crossing_type}. 
\begin{figure}[h]\centering
  \begin{tikzpicture}
    \draw[->] (0,0) node[below] {i} -- (1.5, 2) node[above] {o};
    \fill[white] (0.75, 1) circle (6pt);
    \draw[->] (1.5, 0)--(0, 2) node[above] {u};
    \node at (0.75, -0.5) {$+$};

    \draw[->] (4.5, 0) node[below] {i} --(3, 2) node[above] {o};
    \fill[white] (3.75, 1) circle (6pt);
    \draw[->] (3, 0)--(4.5, 2) node[above] {u};
    \node at (3.75, -0.5) {$-$};
  \end{tikzpicture}
  \caption{Two types of crossing in oriented diagram.\label{crossing_type}}
\end{figure}

Take a commutative ring with unity $R$ and an $R$-module $M$.

\begin{definition}[coloring rule]
  Take $\mathcal{C}\subseteq M^3$ to be a finitely generated submodule of $M^3$. We will call $\mathcal{C}$ a \buff{coloring rule}. There are two submodules $\mathcal{C}_\pm\subseteq \mathcal{C}$, each corresponding to a type of crossing in diagram $D$. 
\end{definition}

We can now construct three homomorphisms
$$\phi:M^3\to M/\mathcal{C}=N$$
$$\phi_\pm:M^3\to M/\mathcal{C}_\pm=N_\pm.$$
We will call $\phi$ and $\mathcal{C}$ \buff{coloring rule} interchangeably.

%Take a commutative ring with unity $R$ and two $R$-modules $M$ and $N$. Take two arbitrary module homomorphisms $\phi_+:M^3\to N$ and $\phi_-:M^3\to N$, one for each type of crossing.

% \begin{definition}[diagram coloring]
%   Let $x_1,..., x_s\in M$ be labels of arcs in diagram $D$. We will say that $(x_1,...,x_s)\in M^s$ is a \buff{coloring} if for every crossing $\pm$ in $D$ consisting of arcs $u$, $i$, $o$ the following relation is satisfied
%   $$\phi_\pm(u,i,o)=0.$$
% \end{definition}

For each crossing $x_j$ in diagram $D$ we can construct a projection 
$$\pi_{x_j}:M^s\twoheadrightarrow M^3$$
which restricts $M^s$ to the three arcs that constitute $x_j$.

\begin{definition}[diagram coloring]
  A \buff{coloring of diagram} $D$ is any element $(m_1,..., m_s)\in M^s$ that assigns elements of $M$ to each arc. We will call this coloring \buff{admissible} if for every crossing $x_j$ of type $\pm$ we have 
  $$\pi_{x_j}(m_1,..., m_s)\in \mathcal{C}_\pm\subseteq\mathcal{C}.$$
\end{definition}

% It is easy to express admissibility of a coloring in terms of homomorphism $\phi$.

It will be beneficial to express admissibility of a coloring in terms of homomorphism $\phi$.
\begin{proposition}
  A coloring $(m_1,..., m_s)\in M^s$ is a admissible $\iff$ for each crossing $x_j$ of type $\pm$ 
  $$\phi_\pm(\pi_{x_j}(m_1,...,m_s))=0.$$
\end{proposition}

\begin{proof}
  Stems from the fact that $\mathcal{C}_\pm=\ker\phi_\pm$.
\end{proof}

% \begin{definition}[diagram coloring]
%   Let $x_1,..., x_s\in M$ be labels of arcs in diagram $D$. We will say that $(x_1,...,x_s)\in M^s$ is a \buff{coloring} if for every crossing $\pm$ in $D$ consisting of arcs $u$, $i$, $o$ the following relation is satisfied
%   $$\phi_\pm(u,i,o)=0.$$
% \end{definition}

% Every crossing in the colored diagram $D$ of knot $K$ yields $x$ relations $\phi_\pm(u,i,o)=0$ which we might treat as linear equations of form 
% $$\phi_+(u,i,o)=au+bi+co=0,$$
% $$\phi_-(u,i,o)=\alpha u+ \beta i+ \gamma o=0,$$
% where $u$, $i$ and $o$ are labels assigned to arcs entering some crossing and $a,b,c\in\Hom(M, N)$.
%
% \begin{definition}
%   Matrix $D\phi:M^s\to N^x$ of coefficients taken from relations $\phi_\pm(u,i,o)$ will be called a \buff{color checking matrix}. 
%   %such that every row has only $3$ nonzero terms, corresponding to arcs entering the appropriate crossing. If $\phi_\pm(u,i,o)=a_\pm u+b_\pm i+c_\pm o$ then those terms will be $a_\pm$, $b_\pm$ and $c_\pm$.
% \end{definition}
%
% Notice that $(x_1,..., x_s)$ is a coloring of the diagram $D$ if and only if it is an element of $\ker D\phi$. However, we can choose $\phi$ to have only a trivial kernel, then only one coloring is admissible - assigning a $0$ to every arc of $D$. Thus, to obtain valuable information about the knot $K$ whose diagram is being colored, we must impose the following restrictions on $\phi$.
% %a knot invariant we must impose some restrictions on $\phi$.

\begin{definition}[color checking matrix]
  After assignings arcs to coordinates in $M^s$ and crossings to coordinates in $N^x$ it is possible to define a linear homomorphism $D\phi:M^s\to N^x$  as
  $$D\phi(m_1,...,m_s)=(\phi_\pm(\pi_{x_1}(m_1,...,m_s)), \phi_\pm(\pi_{x_2}(m_1,...,m_s)),...).$$
  Matrix that is created after choosing a basis for $M^s$ and $N^x$ will be called a \buff{color checking matrix}.
\end{definition}

Taking $\phi_\pm$ to be linear equations of form
$$\phi_+(u,i,o)=au+bi+co$$
$$\phi_-(u,i,o)=\alpha u+\beta i+\gamma o,$$
where $u$, $i$ and $o$ correspond to arcs as seen in \cref{crossing_type} and all the coefficients are linear homomorphisms $M\to N$, we know that all the entries for the color checking matrix will be linear combinations of $a$, $b$, $c$, $\alpha$, $\beta$, $\gamma$. If $M$ has $n$ generators we chose to block the matrix $D\phi$ into $n\times n$ blocks.

\begin{proposition}
  Coloring $(m_1,...,m_s)\in M^s$ is admissible $\iff$ $(m_1,...,m_s)\in\ker D\phi$.
\end{proposition}

\begin{proof}\color{blue}
  $\implies$

  We know that every projection $\pi_{x_j}(m_1,...,m_s)$ is in $\ker\phi_\pm$, depending on the type of $x_j$ crossing. Thus, there is no projection $\pi_{x_j}$ that is not being reduced by $\phi_\pm$.

  $\impliedby$

\end{proof}

{\color{blue}We need to impose restrictions on the coloring rule.} We want $\mathcal{C}$ to be two dimensional (have two generators). That way we have the following diagram
% \begin{center}
% \begin{tikzcd}
%   M^2 & M^3\arrow[l, twoheadrightarrow] \arrow[r, hookleftarrow] & \mathcal{C}\arrow[ll, bend left=20, red, "\sim"]
% \end{tikzcd}
% \end{center}
\begin{center}
\begin{tikzcd}
  M^2 \arrow[r, hookrightarrow] \arrow[rr, bend right=20, red, "\sim" below] & M^3 \arrow[r, twoheadrightarrow] & \mathcal{C}
\end{tikzcd}
\end{center}
We can assume that $M^2$ corresponds to the 'up' and 'in' segments in a crossing (compare \cref{crossing_type}), then we can define $\phi_\pm'$ to take $u$ and $i$ segments and return the out segment so that the labeling agrees with the coloring rule. Now, take the red arrow in the diagram above to be the correspondence
$$(u, i)\mapsto (u, i, \phi_\pm'(u,i)).$$
This demands that both $c$ and $\gamma$ in the definition of $\phi_+$ and $\phi_-$ are invertible. For the sake of simplicity, we will take $c=\gamma=-1$. 

With this assumption for any admissible coloring $(u,i,o)$ of a crossing we have the following relation:
$$\phi_+\;:\;o=au+bi$$
$$\phi_-\;:\;o=\alpha u+\beta i.$$


%it will be beneficial to demand that the red arrow is an isomorphism.
% \begin{enumerate}
%   \item To allow \emph{trivial colorings}, that is colorings in which every arc is assigned the same value it is necessary that
%     $$(\forall\;m\in M)\;\phi_\pm(m,m,m)=0.$$
%   \item To simplify operations of color checking matrices, if
%     $$\phi_+(u, i, o)=au+bi+co$$
%     $$\phi_-(u,i,o)=\alpha u+\beta i+\gamma o,$$
%     then we take $c$ and $\gamma$ to be invertible. For the sake of simplicity, take $c=\gamma=-1$.
%   \item The two variations of orientation of the first Reidemeister move, pictured in \cref{fig: ograniczanie phi reidemeister 1}, put the following constrictions on $a$, $b$ and $\alpha$, $\beta$:
%     $$\begin{cases}
%       a+b=1\\
%       \alpha+\beta=1
%     \end{cases}$$
%   \item Lastly, from the second Reidemeister move, pictured in \cref{fig: ograniczanie phi reidemeister}, one can gather that  
%     $$\begin{cases}
%       a+b\alpha=0\\ 
%       b\beta=1
%     \end{cases}$$
%     meaning that both $b$ and $\beta$ must be units.
% \end{enumerate}
% %
% \begin{figure}[h]\centering
%   \begin{tikzpicture}
%       \begin{knot}[
%         clip width=20pt, 
%         consider self intersections,
%         flip crossing=1
%         ]
%         \strand[thick, ->] (0, 0) to[out=90, in=-90] (0, 1) to[out=90, in=150] (1, 2.5) to[out=-30, in=90] (1.3, 2) to[out=-90, in=30] (1, 1.5) to[out=-150, in=-90] (0, 3) to[out=90, in=-90] (0, 4);
%
%         \strand[thick, ->] (3, 0)--(3, 4);
%
%       \end{knot}
%       \draw[dashed, <->] (1.5, 2)--(2.7, 2);
%
%       \node at (-.5, 2) {$+$};
%       \node at (-1.3, 3.7) {$(a+b)x $};
%       \node at (-.3, -.3) {$x$};
%       \node at (3.3, -.3) {$x$};
%     \begin{scope}[shift={(0, -6)}]
%       \begin{knot}[
%         clip width=20pt, 
%         consider self intersections
%         ]
%         \strand[thick, ->] (0, 0) to[out=90, in=-90] (0, 1) to[out=90, in=150] (1, 2.5) to[out=-30, in=90] (1.3, 2) to[out=-90, in=30] (1, 1.5) to[out=-150, in=-90] (0, 3) to[out=90, in=-90] (0, 4);
%
%         \strand[thick, ->] (3, 0)--(3, 4);
%
%       \end{knot}
%       \draw[dashed, <->] (1.5, 2)--(2.7, 2);
%       \node at (-.5, 2) {$-$};
%
%       \node at (-1.3, 3.7) {$(\alpha+\beta)x $};
%       \node at (-.3, -.3) {$x$};
%       \node at (3.3, -.3) {$x$};
%     \end{scope}
%   \end{tikzpicture}
%   \caption{The two variations of the first Reidemeister move in oriented diagrams. They suggest that $(a+b)=1$ and $(\alpha+\beta)=1$.\label{fig: ograniczanie phi reidemeister 1}}
% \end{figure}
%
% \begin{figure}[h]\centering
%   \begin{tikzpicture}
%     \begin{knot}[
%       clip width=20pt, 
%       flip crossing=1,
%       flip crossing=2
%       ]
%       \strand[thick, ->] (0,-1)--(0, 4);
%       \strand[thick, ->] (-1, -1) to[out=90, in=180+45] (-.8, 0) to[out=45, in=-90] (1, 1.5) to[out=90, in=-45] (-.8, 3) to[out=180-45, in=-90] (-1, 4);
%
%       \strand[thick, ->] (5, -1) -- (5, 4);
%       \strand[thick, ->] (6, -1) -- (6, 4);
%
%     \end{knot}
%     \draw[dashed, <->] (1.5, 1.5)--(4.5, 1.5);
%
%     \node at (-1.3, -.3) {$x$};
%     \node at (.3, -.3) {$y$};
%     \node[anchor=east] at (-.3, 1.5) {$\alpha x+\beta y$};
%     \node[anchor=west] at (.3, 3.7) {$(a+b\alpha) x+b\beta y$};
%
%     \node at (4.7, -.3) {$x$};
%     \node at (6.3, -.3) {$y$};
%
%   \end{tikzpicture}
%   \caption{Second Reidemeister move. It suggests that $(a+b\alpha)=0$ and $b\beta =1$.\label{fig: ograniczanie phi reidemeister}}
% \end{figure}
%
%
% \marginnote{jeszcze przemyśleć tekst}
% It is worth mentioning that examining the second Reidemeister move (\cref{fig: ograniczanie phi reidemeister}) with $\phi_\pm$ changed to $2\times 2$ matrices $A_\pm$, which take arcs entering a crossing as input and output the arcs leaving it, we can see that 
% $$A_+ A_-=\begin{bmatrix}0 & 1\\ b & a\end{bmatrix}\begin{bmatrix}\alpha & \beta \\ 1 & 0\end{bmatrix}=Id.$$
% This means that from homomorphism $\phi_+$ we are able to calculate $\phi_-$ and vice versa.
%
% The color checking matrix is not a knot invariant, despite the restrictions laid on $\phi$. Changing the number of crossings in a diagram will obviously create a different matrix for the same knot. We will thus proceed to define an equivalence relation on the set of all color checking matrices of a knot $K$.

% Let $R$ be a commutative ring, typically $\Z[\Z]$, and take $M,N$ to be two $R$-modules. Consider two module homomorphisms $\phi_+:M^3\to N$ and $\phi_-:M^3\to N$ such that
% $$(\forall\;x\in M)\;\phi_\pm(x,x,x)=0.$$ 
% This homomorphism will be used to determine whether or not a labelling of knot arcs constitutes a coloring or not.
%
% \medskip
%
%
% {\large\color{red}tak naprawdę wystarczy jeden homomorfizm $\phi$, ale nie wiem jeszcze jak to wyjaśnić beż zahaczania o linki lub warkocze}.
% %{\large\color{red}JAK WUTLUMACZYC, ZE NAPRAWDE TO WYSTARCZY JEDNA FUNKCJA, ALE TAK BEDZIE MI LATWIEJ W ZYCIU? BO JAK NARYSUJĘ DIAGRAM Z DWOMA SKRZYŻOWANIAMI JEDNO + A DRUGIE - I POŁĄCZĘ JAK PRZY WARKOCZE -> LINKI TO DOSTAJĘ LINKA Z 2 KOMPONENTAMI :V}
% % It might appear that in the case of oriented diagram two different functions $\phi$ are required. However, considering the following diagram
% % \begin{figure}[h]\centering 
% %   \begin{tikzpicture}
% %     \draw[->] (0,-0.5) to (0, -1) to [out=-90, in=90] (1, -3);
% %     \draw[->] (1, -3) to (1, -3.5) to[out=-90, in=90] (0, -5.5) to (0,-6);
% %
% %     \fill[white] (.5, -2) circle (6pt);
% %     \fill[white] (.5, -4.5) circle (6pt);
% %
% %     \draw[->] (1,-0.5) to (1, -1) to[out=-90, in=90] (0, -3);
% %     \draw[->] (0, -3) to (0, -3.5) to[out=-90, in=90] (1, -5.5) to (1, -6);
% %     \draw (1, -6) to [out=-90, in=-90] (3, -4) to[out=90, in=-90] (3, -2.5) to[out=90, in=90] (1, -.5);
% %
% %     \node at (1, -2) {$-$};
% %     \node at (1, -4.5) {$+$};
% %   \end{tikzpicture}
% %   \caption{Why only one function $\phi$ is required?\label{just_why_phi}}
% % \end{figure}
%
% \medskip
%
%
% The color checking matrix in itself is obviously not a knot invariant. However, we might define an equivalence relation on the set of all matrices $M^m\to N^n$, $m,n\in\N$, such that all the matrices which come from the same knot fall into the same equivalence class.
%



We might also demand that a trivial coloring (every arc is assigned the same element of $M$) is an admissible coloring.

The color checking matrix is not a knot invariant. Changing the diagram with accordance to the Reidemeister moves might change the dimensions of the matrix. Thus, we need to define an equivalence relation on the set of all color checking matrices.

