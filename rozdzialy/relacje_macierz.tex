\section{What is a knot coloring}

Let $K$ be a knot and $D$ be its oriented diagram with $s$ segments and $x$ crossings. In such diagrams we can see two different crossing types as seen in \cref{crossing_type}.
\begin{figure}[h]\centering
  \begin{tikzpicture}
    \draw[->] (0,0) -- (1.5, 2);
    \fill[white] (0.75, 1) circle (6pt);
    \draw[->] (1.5, 0)--(0, 2);
    \node at (0.75, -0.5) {$+$};

    \draw[->] (4.5, 0)--(3, 2);
    \fill[white] (3.75, 1) circle (6pt);
    \draw[->] (3, 0)--(4.5, 2);
    \node at (3.75, -0.5) {$-$};
  \end{tikzpicture}
  \caption{Two types of crossing in oriented diagram.\label{crossing_type}}
\end{figure}

Let $R$ be a commutative ring, typically $\Z[\Z]$, and take $M,N$ to be two $R$-modules. Consider two module homomorphisms $\phi_+:M^3\to N$ and $\phi_-:M^3\to N$ such that
$$(\forall\;x\in M)\;\phi_\pm(x,x,x)=0.$$ 
This homomorphism will be used to determine whether or not a labelling of knot arcs constitutes a coloring or not.

\medskip

\begin{definition}[diagram coloring]
  Let $x_1,..., x_s\in M$ be labels of arcs in diagram $D$. We will say that $(x_1,...,x_s)\in M^s$ is a \textbf{coloring} if for every crossing $\pm$ in $D$ consisting of arcs $u$, $i$, $o$ the following relation is satisfied
  $$\phi_\pm(u,i,o)=0.$$
\end{definition}

{\large\color{red}tak naprawdę wystarczy jeden homomorfizm $\phi$, ale nie wiem jeszcze jak to wyjaśnić beż zahaczania o linki lub warkocze}.
%{\large\color{red}JAK WUTLUMACZYC, ZE NAPRAWDE TO WYSTARCZY JEDNA FUNKCJA, ALE TAK BEDZIE MI LATWIEJ W ZYCIU? BO JAK NARYSUJĘ DIAGRAM Z DWOMA SKRZYŻOWANIAMI JEDNO + A DRUGIE - I POŁĄCZĘ JAK PRZY WARKOCZE -> LINKI TO DOSTAJĘ LINKA Z 2 KOMPONENTAMI :V}
% It might appear that in the case of oriented diagram two different functions $\phi$ are required. However, considering the following diagram
% \begin{figure}[h]\centering 
%   \begin{tikzpicture}
%     \draw[->] (0,-0.5) to (0, -1) to [out=-90, in=90] (1, -3);
%     \draw[->] (1, -3) to (1, -3.5) to[out=-90, in=90] (0, -5.5) to (0,-6);
%
%     \fill[white] (.5, -2) circle (6pt);
%     \fill[white] (.5, -4.5) circle (6pt);
%
%     \draw[->] (1,-0.5) to (1, -1) to[out=-90, in=90] (0, -3);
%     \draw[->] (0, -3) to (0, -3.5) to[out=-90, in=90] (1, -5.5) to (1, -6);
%     \draw (1, -6) to [out=-90, in=-90] (3, -4) to[out=90, in=-90] (3, -2.5) to[out=90, in=90] (1, -.5);
%
%     \node at (1, -2) {$-$};
%     \node at (1, -4.5) {$+$};
%   \end{tikzpicture}
%   \caption{Why only one function $\phi$ is required?\label{just_why_phi}}
% \end{figure}

Every crossing in the diagram $D$ of knot $K$ yields $x$ relations $\phi_\pm(u,i,o)=0$ which we might treat as linear equations of form 
$$\phi_\pm(u,i,o)=au+bi+co=0,$$
where the $s$ arcs act as variables and $\color{red}a+b+c=0\in \Hom(M, N)$ (when $M=N$ then $a+b+c\in\Ann(M)$).

\medskip

\begin{definition}
  Matrix $D\phi:M^s\to N^x$ of coefficients taken from relations $\phi_\pm(u,i,o)=0$ will be called a \textbf{color checking matrix}. 
  %such that every row has only $3$ nonzero terms, corresponding to arcs entering the appropriate crossing. If $\phi_\pm(u,i,o)=a_\pm u+b_\pm i+c_\pm o$ then those terms will be $a_\pm$, $b_\pm$ and $c_\pm$.
\end{definition}

The color checking matrix in itself is obviously not a knot invariant. However, we might define an equivalence relation on the set of all matrices $M^m\to N^n$, $m,n\in\N$, such that all the matrices which come from the same knot fall into the same equivalence class.


