%In reality, the equivalence between diagrams which relates diagrams of the same knot is much more subtle than the isomorphism from the previous section. This relation $R$ is generated by the Reidemeister moves (see \cref{reidemeister-generating}). There is an induced relation on the set of color checking matrices, called $D(R)$, that relates color checking matrices of the same knot. 
%\hl{For suitably chosen palettes, an invariant of the equivalent class of matrices of the same knot is known.} 

In this section we will define a palette for which the assigment $D\mapsto D\phi$ has all the properties of a functor.

Consider a crossing
\begin{center}
  \begin{tikzpicture}
    \draw[<-] (0, 0) node[above] {$o$} --(2, 0) node[above] {$i$};
    \fill[white](1, 0) circle (7pt);
    \draw[->] (1, -1) node[right] {$u$}--(1, 1);
  \end{tikzpicture}
\end{center}
and take some $x$ to be the generator in the Wirtinger presentation of the knot group that is used to generate a representation for $K_G^{ab}$ (see \cref{alexander module discussion}). Then, the following is a relation in said group
$$UxCx(Ux)^{-1}=Ix$$
where $U=ux^{-1}$, $I=ix^{-1}$ and $O=ox^{-1}$. 
We can multiply both sides by $x^{-1}$ to obtain
$$x^{-1}UxCU^{-1}=x^{-1}Ix$$
which is change in $\Z[\Z]$ to
$$
tU+C-U=tI\implies 0=(1-t)U+tI-C
$$
The procedure for the other type of crossing is analogous and yields relation $0=(1-t^{-1})U+t^{-1}I-C$.

\begin{definition}[Alexander palette]
A palette {\boldmath$(R=\Z[\Z], M=\Z[\Z], \mathcal{C}_\pm)$}, where $\mathcal{C}_\pm$ are defined by homomorphisms
$$\phi_+(u,i,o)=(1-t)u+ti-o$$
$$\phi_-(u,i,o)=(1-t^{-1})u+t^{-1}i-o,$$
with $u$, $i$ and $o$ defined in \cref{crossing_type}, is called the \buff{Alexander palette}.
\end{definition}


% Thus, the equivalence class of color checking matrices of a knot whose diagrams are being colored with the Alexander palette will be the same as for the Alexander matrix. Thus, the determinant up to multiplication of a unit of a minor of the color checking matrix will be a knot invariant.


The following example illustrates the importance of choosing a suitable palette.

\begin{example}
  Consider a coloring of trefoil knot $3_1$ with two palettes: $P_1=(\Z, \Z, \phi_\pm(u,i,o)=2u-i+o)$ that is not an image of the Alexander palette and $P_2=(\Z, \Z, \phi_\pm(u,i,o)=2u-i-o)$ which in turn is one (by $\Z[\Z]\ni t\mapsto -1\in\Z$). On the diagram below the color checking matrices of the diagram with $3$ crossings along with a set of $2\times 2$ minor values (on the right of the matrix) of it are presented
  \begin{center}
    \begin{tikzpicture}
      \begin{knot}[
        consider self intersections, 
        clip width = 20pt, 
        % draft mode=crossings, 
        flip crossing=2
        ] 
        \strand[->, thick] (90:2) to[out=0, in=-60, looseness=2]
        (210:2) to[out=120, in=60, looseness=2]
        (-30:2) to[out=-120, in=180, looseness=2] 
        (90:2);
      \end{knot}

      \node[anchor=west] at (3, 1.8) {
        $P_1(D)=
        \begin{bmatrix}
          2 & -1 & 1\\ 
          -1 & 1 & 2 \\ 
          1 & 2 & -1 
        \end{bmatrix}\mapsto \{1, -3, -5\} 
      $};

    \node[anchor=west] (p2) at (3, -1.2) {
        $P_2(D)=
        \begin{bmatrix}
          2 & -1 & -1\\ 
          -1 & -1 & 2 \\ 
          -1 & 2 & -1 
        \end{bmatrix}\mapsto\{-3\}
      $};
    \end{tikzpicture}
  \end{center}

  while after the first Reidemeister move, the color checking matrix is

  \begin{center}
    \begin{tikzpicture}
      \begin{knot}[
        consider self intersections, 
        clip width = 20pt,
        %ignore endpoint intersections=false,
        % draft mode=crossings,
        % flip crossin=1,
        % flip crossing=2
        ] 
        \strand[<-, thick] (80:3) to[out=-100, in=30] 
        (90:2) to[out=-30, in=-60, looseness=2]
        (210:2) to[out=120, in=60, looseness=2]
        (-30:2) to[out=-120, in=210, looseness=2] 
        (90:2) to[out=150, in=-80] 
        (100:3) to[out=100, in=80, looseness=2] 
        (80:3);
      \end{knot}
      \fill[white] (90:2) circle (10pt);
      \draw[thick] ($(90:2)+({0.3*cos(210)}, {0.3*sin(210)})+(-0.02, -0.07)$) to[out=30, in=-120] ($(90:2) + ({0.3*cos(30}, {0.3*sin(30})+(0, 0.08)$);%(80:3);
      % \draw[->, thin] (90:2) to[out=-30, in=-60, loosenes=2] (210:2);
      
      \node[anchor=west] at (3, 2.5) {$P_1(D)=
          \begin{bmatrix}
            0 & 2 & 1 & -1 \\ 
            0 & -1 & 2 & 1 \\ 
            1 & 0 & -1 & 2\\ 
            1 & 1 & 0 & 0
          \end{bmatrix}\mapsto\{5, 11, 3, ...\}$
        };
      
     
      \node[anchor=west] at (3, -1) {$P_2(D)=
          \begin{bmatrix}
            0 & 2 & -1 & -1 \\ 
            0 & -1 & 2 & -1 \\ 
            -1 & 0 & -1 & 2\\ 
            1 & -1 & 0 & 0
          \end{bmatrix}\mapsto\{-3, 3\}$
        };
    \end{tikzpicture}
  \end{center}
\end{example}

% In order to ensure that the equivalence classes of color checking matrices under the relation induced by Reidemeister moves on diagrams have a known invariant, rules must be imposed on palettes. A particular set of palettes that will satisfy conditions to be mention are the Alexander palette and all palettes derived from it by either a ring homomorphism or a module homomorphism.

For any palette for which $D\mapsto D\phi$ is a functor, in particular the Alexander palette, 
the coloring rule modules $\mathcal{C}_\pm$ are isomorphic to $M^2$
    $$
    \begin{tikzcd}
      M^2 \arrow[r, hookrightarrow] \arrow[rr, bend right=25, red, "\cong" below] & M^3 \arrow[twoheadrightarrow, r] & \mathcal{C}_\pm.
    \end{tikzcd}
    $$
The red isomorphism to be $(u, i)\mapsto (u, i, \phi_\pm'(u, i))\in\mathcal{C}_\pm$, with segments labeled like in \cref{crossing_type}, meaning that $c$ and $\gamma$ in \eqref{phi equations1} and \eqref{phi equations2} respectively are units. For the sake of convenience, take $c=\gamma=-1$.

    This property of palettes will be called a \buff{propagation rule} as knowing colors of two of the three segments allows one to calculate the color assigned to the remaining segment.

    \begin{lemma}\label{warunki na palete}
  A palette that meets the propagation rule has the following two properties determined by Reidemeister moves.
  \begin{enumerate}
    \item The first Reidemeister move requires that
      $$a=1-b$$
      $$\alpha=1-\beta,$$
      where the variables are coefficients from \eqref{phi equations1} and \eqref{phi equations2}.  
    \item Similarly, the second Reidemeister move requires
      $$\begin{cases}
        a\beta+\alpha=0\\ 
        \beta b=1.
      \end{cases}$$
  \end{enumerate}
\end{lemma}

Proof of the lemma is divided into two parts, each given in the next session after defining the relation induced by the Reidemeister moves on color checking matrices.
