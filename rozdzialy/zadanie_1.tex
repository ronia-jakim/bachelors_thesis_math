%\section{Problem}

%{\bfseries%
%  Consider the ring $\Z[[F]$, where $[F]$ is the equivalence class of all finite Abelian groups isomorphic to $F$. Describe the set $\Z[[F]]/\{[F_2]=[F_1]+[F_3]\}$, where relation $[F_2]=[F_1]+[F_3]$ means that there exists exact sequence:
%
%  \begin{center}\begin{tikzcd}
%    0\arrow[r] & F_1\arrow[r] & F_2\arrow[r] & F_3\arrow[r] & 0
%  \end{tikzcd}\end{center}
%}

\section{What does $\Z_2\oplus\Z_2$ and $\Z_4$ have in common? (problem 1)}

Consider the group $\Z[[F]]$, where $[F]$ is the equivalence class of all finite Abelian groups isomorphic to $F$. Let $\heartsuit$ be the equivalence closure of relation defined as follows: if $F_1,F_2,F_3$ are Abelian groups the $[F_2]=[F_1]+[F_3]$ if there exists an exact sequence
\begin{center}\begin{tikzcd}
  0\arrow[r] & F_1\arrow[r] & F_2\arrow[r] & F_3\arrow[r] & 0
\end{tikzcd}\end{center}

\begin{lemma}\label{[F]=[F']}
  If $F, F'$ are two Abelian groups of order $n$, then they represent the same equivalence class of relation $ \heartsuit $ i.e. $[F]_\heartsuit = [F']_\heartsuit$.
\end{lemma}

\begin{example}\label{Z2+Z2=Z4}
  Before we prove \cref{[F]=[F']}, let us examine an example. We will show that $[\Z_4]=[\Z_2\oplus\Z_2]$. Consider the following exact sequence
  \begin{center}\begin{tikzcd}[column sep=large]
      0 \arrow[r] & \Z_2 \arrow[r, "\times 2"] & \Z_4 \arrow[r, "\mod 2"] & \Z_2 \arrow[r] & 0
  \end{tikzcd}\end{center}
  which shows that $[\Z_4]=[\Z_2]+[\Z_2]$. On the other hand, the next sequence
  \begin{center}\begin{tikzcd}[column sep=large]
      0 \arrow[r] & \Z_2 \arrow[r, "i_1"] & \Z_2\oplus\Z_2 \arrow[r, "\pi"] & \Z_2 \arrow[r] & 0
  \end{tikzcd}\end{center}
  which is also exact, yields $[\Z_2\oplus\Z_2]=[\Z_2]+[\Z_2]$.

  This shows that every Abelian group of order $4$ is in the same equivalence class of relation given by exact sequences. We will show that all Abelian groups of the same order will belong to one equivalence class.
\end{example}

\begin{proof}
Every finite Abelian group is isomorphic to a direct product of its $p$-subgroups{\large\color{red}DODAĆ CYTAT}. Furthermore, any $p$-group of order $p^k$ is isomorphic to $\Z_{p^k}$. We can start by examining what elements belong to equivalence class $[\Z_{p^k}]$.

We will start by showing that if $k=n+l,\; k,n,l\in\N$, then $[\Z_{p^k}]=[\Z_{p^n}]+[\Z_{p^l}]$. Consider the exact sequence

\begin{center}\begin{tikzcd}
  0 \arrow[r] & \Z_{p^n} \arrow[r] & \Z_{p^k} \arrow[r] & \Z_{p^k}/\Z_{p^n}\cong \Z_{p^l} \arrow[r] & 0
\end{tikzcd}\end{center}

We know that $\Z_{p^k}/\Z_{p^n}$ is a cyclic group generated by $1+Z_{p^n}$. Furthermore, we know that $|\Z_{p^k}/\Z_{p^n}|=p^l$ and thus $\Z_{p^k}/\Z_{p^n}\cong \Z_{p^l}$.


%Define $f(1)=p^l\mod p^k$ and $g(1)=1\mod l$. We now need to check if $\ker g=\im f$. Take any $x\in\ker g$, then $x=l\cdot m\mod n$ for some $m\in\{0, 1, 2, ..., k-1\}$. Then for $m\mod l\in\Z_l$ we have $f(m\mod l)=ml\mod n$ and so $x\in\ker g\iff x\in \im f$. This shows that the sequence above is exact and $[\Z_n]=[\Z_k]+[\Z_l]$.

Now, we will show, using induction on $N$, that for any $n\in\N$ such that $n=\prod_{i=1}^N p_i^{k_i}$, where $k_i\in\N$ and $p_i$ is a prime number, we have
$$ [\Z_n] = \sum_{i=1}^N[\Z_{p_i^{k_i}}] = \sum_{i=1}^N k_i\cdot[\Z_{p_i}]\quad (\star) $$

\begin{enumerate}
  \item $N=1$

    From the fact above we know that $[\Z_{p^{k+1}}]=[\Z_{p^k}]+[\Z_p]$ and applying the same reasoning to $\Z_{p^k}$ we obtain $[\Z_{p^k}]=k\cdot[\Z_p]$.
  \item $N-1\implies N$

    We will start from the right side of the equality $(\star)$ and from inductive hypothesis we know that
    $$\sum_{i=1}^{N}k_i[\Z_{p_i}]=k_{N}[\Z_{p_{N}}]+\sum_{i=1}^{N-1}k_i[\Z_{p_i}]=[\Z_{p_{N}^{k_{N}}}]+[\Z_l]$$
    where $l=\prod_{i=1}^{N-1}p_i^{k_i}$. Consider the following sequence
    \begin{center}\begin{tikzcd}
      0 \arrow[r] & \Z_l \arrow[r] & \Z_n \arrow[r] & \Z_{p_N^{k_N}} \arrow[r] & 0
    \end{tikzcd}\end{center}
    its exactness follows from the fact that $\Z_n/\Z_l$ is a cyclic group of order $n/l=p_N^{k_N}$ and thus there exists an isomorphism
    $$\Z_{p_N^{k_N}}\cong \Z_n/\Z_l.$$
\end{enumerate}

As stated before, any Abelian group of order $N$ is isomorphic to a direct product of its $p$-subgroups, hence the following equality is immediate from $(\star)$:
$$\sum_{i=1}^N k_i[\Z_{p_i}] = [\Z_n]=\left[ \bigoplus_{i=1}^N \Z_{p_i^{k_i}} \right]$$
\end{proof}

From this follows that every Abelian group of order $n$, either being a cyclic group itself or a direct sum of cyclic groups, is in one equivalence class. Hence, elements of group
$$\Z[[F]]/\{[F_2]=[F_1]+[F_3]\}$$
can be expressed as finite sums of equivalence classes represented by $p$-groups:
$$\Z[[F]/\{[F_2]=[F_1]+[F_3]\}=\{\sum_{i\leq n} k_i[\Z_{p_i}]\;:\;p_i\text{ are prime}, n,k_i\in\N\}$$
