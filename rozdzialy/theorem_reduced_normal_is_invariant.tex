\begin{theorem}
  The reduced normal form of color checking matrix using the Alexander palette does not depend on the choice of diagram $D$ of a knot $K$. Thus, it is well defined for $K\phi$ and is a knot invariant.
\end{theorem}

\begin{proof}
  % \marginnote{nie wiem, czy tutaj aż tak powinno się dokładnie mówić co i jak dodaję?} % TO DO
  Take a knot $K$ and its diagram $D$ with $s$ segments and $x$ crossings ($s=x$ by \cref{ilosc segmentow to ilosc skrzyzowan}). We will start by showing that applying any Reidemeister move to obtain a new diagram $D'$ will not change the reduced normal form of its color checking matrix.

  \subsection*{\centering R1}

  The first Reidemeister move is split into \textbf{R1a} and \textbf{R1b}. Due to those two cases being analogous, we will focus on the move \textbf{R1a}.

  Take $D'$ to be diagram $D$ with one arc twisted into a $+$ crossing. In the previous section, due to the relation $D(R1a)$, the matrices $D\phi$ and $D'\phi$ are as follows
  $$
  D'\phi=
  \begin{bmatrix}
    b & a-1 & 0 & \hdots\\ 
    x_2 & y_2 & \hdots \\ 
    x_3 & y_3 \\ 
    \vdots 
  \end{bmatrix}
  $$
  $$
  D\phi=
  \begin{bmatrix}
    x_2 + y_2 & \hdots \\ 
    x_3 + y_3 \\ 
    \vdots
  \end{bmatrix}
  $$
  with $(\forall\; i\geq 2)\;x_i=0\;\lor\; y_i=0$. Adding the first column of $D'\phi$ to the second column will yield 
  $$
  D'\phi=
  \begin{bmatrix}
    b & 0 & 0 & \hdots\\ 
    x_2 & x_2+y_2 & \hdots \\ 
    x_3 & x_3+y_3 \\ 
    \vdots 
  \end{bmatrix}
  $$
  because $a+b=1$. Now we know that $b$ is a unit, thus we can easily remove the elements of the first column that are not $b$. This results in 
  $$
  D'\phi=
  \begin{bmatrix}
    b & 0 & 0 & \hdots\\ 
    0 & x_2+y_2 & \hdots \\ 
    0 & x_3+y_3 \\ 
    \vdots 
  \end{bmatrix}
  $$
  notice that the lower right portion of this matrix looks exactly like $D\phi$. The only difference is a column containing a singular unit element and thus it will be struck out when computing the reduced normal form. Therefore, the reduced normal form of $D'\phi$ is the same as in $D\phi$.
  
  \subsection*{\centering R2}

  Now the diagram $D'$ is a diagram $D$ with one arc poked onto another. Once again 
  %we will put those changed arcs at the beggining of 
  the color checking matrices are:
  $$
  D'\phi=
  \begin{bmatrix}
    \alpha & \beta & -1 & 0 & \hdots \\ 
    a & 0 & b & -1  \\ 
    x_3 & u_3 & 0 & v_3 \\ 
    x_4 & u_4 & 0 & v_4 \\ 
    \vdots & & & & \ddots
  \end{bmatrix}
  $$
  $$
  D\phi= 
  \begin{bmatrix}
    x_3 & u_3 + v_3 & \hdots \\ 
    x_4 & u_4 + v_4 \\ 
    \vdots
  \end{bmatrix}
  $$
  with $(\forall\;i\geq 3)\;u_3=0\;\lor\;v_3=0$. Adding the third column of $D'\phi$ multiplied by $\alpha$ and $\beta$ to first and second column respectively we are able to reduce the first row to only zeros and $-1$. Now, adding this row to the second one creates a column with only $-1$ and zeros. We can put it as the first column:
  $$
  D'\phi=
  \begin{bmatrix}
    -1 & 0 & 0 & 0 & \hdots \\ 
    0 & a +b\alpha & 0 & -1  \\ 
    0 & x_3 & u_3 & v_3 \\ 
    0 & x_4 & u_4 & v_4 \\ 
    \vdots & & & & \ddots
  \end{bmatrix}
  $$
  Notice that $a+b\alpha=0$ and so we can transform this matrix into
  $$
  D'\phi=
  \begin{bmatrix}
    -1 & 0 & 0 & 0 & \hdots \\ 
    0 & -1 & -1 & 0  \\ 
    0 & v_3 +u_3 & v_3+u_3&  x_3 & \\ 
    0 & v_4 +u_4 & v_4+u_4 & x_4 \\ 
    \vdots & & & & \ddots
  \end{bmatrix}
  $$
  and then into 
  $$
  D'\phi=
  \begin{bmatrix}
    -1 & 0 & 0 & 0 & \hdots \\ 
    0 & -1 & 0 & 0  \\ 
    0 & 0 & v_3+u_3&  x_3 & \\ 
    0 & 0 & v_4+u_4 & x_4 \\ 
    \vdots & & & & \ddots
  \end{bmatrix}
  $$
  which obviously has the same reduced normal form as $D\phi$.

  \subsection*{\centering R3}

  The last Reidemeister move creates the following two matrices
  $$D\phi=
  \begin{bmatrix}
    \alpha & -1 & \beta & 0 & 0 & 0 \\ 
    0 & 0 & -1 & b & 0 & a \\ 
    \beta & 0 & 0 & 0 & -1 & \alpha \\ 
    u_4 & 0 & v_4 & w_4 & x_4 & y_4
  \end{bmatrix}
  $$
  $$D'\phi=
  \begin{bmatrix}
    0 & 0 & -1 & \beta & \alpha & 0 \\ 
    \beta & 0 & 0 & 0 & -1 & \alpha \\ 
    0 & -1 & b & 0 & 0 & a\\ 
    u_4 & 0 & v_4 & w_4 & x_4 & y_4
  \end{bmatrix}
  $$

  Applying row and column operations on those matrices results in 
  $$
  D\phi=
  \begin{bmatrix}
    -1 & 0 & 0 & 0 & 0 & 0 \\
    0 & -1 & 0 & 0 & 0 & 0 \\ 
    0 & 0 & \beta & 0 & -1 & 0 \\ 
    0 & 0 & u_4+v_4 & w_4+v_4 & x_4-v_4 & y_4+u_4+x_4
  \end{bmatrix}
  $$
  $$
  D'\phi=
  \begin{bmatrix}
    b & 0 & 0 & 0 & 0 & 0 \\
    0 & -\beta & 0 & 0 & 0 & 0 \\ 
    0 & 0 & \beta & 0 & -1 & 0 \\ 
    0 & 0 & u_4+v_4 & w_4+v_4 & x_4-v_4 & y_4+u_4+x_4
  \end{bmatrix}
  $$
  which makes clear that those matrices have the same reduced normal form as $b$ and $\beta$ were taken to be units.

  Notice that if $A\sim B$ and $B\sim C$, where $\sim$ means having the same reduced Smith normal form, then $A\sim C$. Thus, if two knots differ by a finite sequence of Reidemeister moves (as is the case for different diagrams of the same knot), then their reduced Smith normal forms are equal.

\end{proof}
