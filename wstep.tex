\section*{Abstract}

The knot group $G=\pi_1(K)$ is a starting point for many knot invariants. Alexander matrix is a representation matrix for a subgroup of $G$ and from its determinant, the Alexander polynomial is obtained. Another way of obtaining said polynomial is by considering a coloring matrix which assigns elements of $R$-module $M$ to segments from a diagram $D$ of knot $K$. This approach can be derived from the image of a resolution of Alexander module through the functor $\Hom(-, M)$. Nevertheless, color checking matrices do not instantly yield a knot invariant, however it is possible to define an equivalence relation that identifies matrices stemming from the same knot. This approach is used to distinguish a pair of knots with the same Alexander polynomial. In the end, a way of generalizing the procedure of coloring diagrams is presented in terms of category theory.

\section*{Introduction}

In knot theory distinguishing knots is often a difficult endeavor, usually facilitated by the notion of invariants. An interesting group of knot invariants are polynomial invariants, such as the Alexander polynomial. Another group worth mentioning are knot colorings that can also yield an element of the ring $\Z[\Z]$.

Very often, considering only one invariant is not sufficient, as there are many knots that share its value, i.e. $K11n85$ and $K11n164$ have the same Alexander polynomial. However, a more subtle application of the same method that yields the Alexander polynomial can sometimes distinguish such knots.

The following paper is a result of a year long cooperation between prof. Tadeusz Januszkiewicz, Julia Walczuk and the author of this thesis. In it, connections between the knot group, knot colorings and homology modules of infinite cyclic covering (see \cite{milnor_infinite_cyclic}) will be outlined. As an additional exercise, we will show a way of distinguishing already mentioned knots $K11n85$ and $K11n164$.

The first section of this paper defines the most important terms used in knot theory, as well as highlights the connection between the metabelianization of knot group and the first homology module of an infinite cyclic covering of the complement of said knot. 

Subsequently, the construction of a $\Z[\Z]$ module from the kernel of $G^m\to \Z$ is presented. This module is defined to be the Alexander module $K_G^{ab}$ of knot $K$ (\cref{alexander module def}). Then, Alexander matrix is introduced (\cref{alexander matrixi def}) as the representation matrix of the Alexander module. In this section the resolution of Alexander module is proven to be of form  
\begin{center}
  \begin{tikzcd}
    0\arrow[r] & \Z[\Z]\arrow[r] & \Z[\Z]^{n}\arrow[r] & \Z[\Z]^{n-1}\arrow[r] & K_G^{ab}\arrow[r] & 0 
  \end{tikzcd}
\end{center}
At the end of the second section, a connection between resolution of the Alexander module and coloring of diagrams is made.

The third and last section is focused on coloring matrices (\cref{def:color checking matrix}) and defining an equivalence relation between matrices relating to the same knot. A new knot invariant is defined (\cref{reduced normal form def}) and an example of its utility is presented in \cref{example reduced normal form}.

Last section is dedicated to presenting an approach to diagram colorings from the perspective from category theory. In addition, a connection between the Alexander matrix and color checking matrix (using a particular palette, named the Alexander palette) is presented.
